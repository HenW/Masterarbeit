\chapter{Einleitung}\label{ch:einleitung}

Die Technologieentwicklung, im speziellen die der Computer, kann auf eine rasante Entwicklung in den letzten
Jahrzehnten zurückblicken. Jedoch ist diese Technik mittlerweile so hochentwickelt, dass es mit zunehmender Zeit
schwieriger wird, die Bauteile noch schneller, kleiner und effizienter herzustellen, denn auch hier gibt es Grenzen.
So liegt es an den Entwicklern neue Wege zu finden, um diese Beschränkungen zu umgehen. \\
So kam es zur Entwicklung der \acp{fpga}, welche es nun ermöglichen sollten, komplexe Logikschaltungen
rekonfigurierbar herzustellen, sodass diese jederzeit verändert und damit den Gegebenheiten und
Anforderungen angepasst werden konnten. Durch die Optimierung während der Laufzeit können Kosten gesenkt und
Ressourcen geschont werden. Mittlerweile sind moderne \acp{fpga} mit genügend Recourcen ausgestattet, um eine
große Anzahl an Logikgattern nachzubilden und sind in der Lage komplexe Rechenoperationen durchzuführen.\\
Aus dieser Entwicklung resultiert die Entstehung eingebetteter Systeme auf Basis von \acp{fpga}, welche mittlerweile eine hohe Anzahl an Aufgaben in allen Bereichen der Technik übernehmen.
Sie bieten eine hohe Anpassungsfähigkeit bei gleichzeitig gut überschaubaren Kosten für Herstellung und Entwicklung.\\

\section{Ziel der Arbeit}\label{kap:zielderarbeit}

Das Ziel dieser Arbeit ist es, verschiedene solcher \ac{fpga}-basierter, eingebettete Systeme
auf dem Digilent Nexys4 DDR \ac{fpga}-Board identisch zu konfigurieren und verschiedene Benchmarks auszuführen.
Die ausgewählten Systeme für diese Arbeit sind der LEON3 von Cobham Gaisler, der Microblaze von Xilinx,
der lowRISC, sowie das \ac{prhs}, ein System der
Professur für Technische Informatik an der Helmut-Schmidt-Universität, Universität der Bundeswehr in Hamburg.\\
\newpage
Für die Bewertung soll auf der einen Seite der Verbrauch von Ressourcen auf dem \ac{fpga}-Board betrachtet und
auf der anderen Seite leistungsspezifische Benchmarks
wie Coremark, Dhrystone und Ramspeed ausgeführt werden. Die erzielten Ergebnisse der verschiedenen Benchmarks sollen
grafisch dargestellt werden, um so einen direkten Vergleich der Systeme zu vereinfachen. Des Weiteren folgt anhand
 dieser Ergebnisse ein Fazit und eine
Bewertung der Systeme.\\


 \section{Struktur der Arbeit}\label{kap:strukturderarbeit}

Zur Beschreibung des Vorgehens, wird die Arbeit wie folgt gegliedert.\\
 Zu Beginn der Arbeit werden im Kapitel~\ref{kap:grundlagen} die theoretischen Grundlagen vermittelt. Als
Ausgangspunkt wird im Kapitel~\ref{kap:nexys4} auf das, in dieser Arbeit, verwendete \ac{fpga}-Board, sowie auf den sich darauf befindenden \ac{fpga} (Kapitel~\ref{kap:fpga}) eingegangen.
 Um die Grundlagen der eingebetteten Systeme zu erklären, werden im Kapitel~\ref{kap:eingebettetesysteme} die Definition, sowie verschiedene Arten von eingebetteten Systemen erläutert, gefolgt von
 einem Vergleich der Vor- und Nachteile des Linux für Embedded Systems in Kapitel~\ref{kap:vorundnachteilelinux}. Das Betriebssystem Linux wird im darauffolgenden Kapitel~\ref{kap:softwarevoraussetzung} näher
 erläutert, unter dem Aspekt der Softwarevoraussetzungen für Embedded Systems, nachdem in Kapitel~\ref{kap:voraussetzungen} die minimal Anforderungen für ein eingebettetes System kurz dargestellt wurden.
 Ebenfalls deckt das Kapitel~\ref{kap:softwarevoraussetzung} das Thema \emph{Compiler} ab, welches gerade im Bezug zum \emph{Cross-Compiling} der Benchmarks ein wichtiger Bestandteil ist.
 In Kapitel~\ref{kap:hardwarevoraussetzung} folgt eine umfangreiche Erklärung zu den benötigten Hardwarekomponenten für eingebettete Systeme und deren Zusammenspiel. Dazu zählen neben der \ac{cpu}, hier im speziellen die
 Softcore-Prozessoren und deren Vergleich mit Hardcore-Prozessoren, unter anderem auch die \ac{mmu}, der Cache, sowie die Erläuterung von Interrupts, Timern und einigen Peripheriegeräten. Um die Leistung
 der ausgewählten Systeme zu testen und zu vergleichen, werden im Kapitel~\ref{kap:benchmark} zunächst Benchmarks grundsätzlich erklärt, woraufhin in den nachfolgenden Kapiteln~\ref{kap:dhrystone},
\ref{kap:coremark} und~\ref{kap:ramspeed} die einzelnen Benchmarks nochmals in ihrer Funktionsweise näher erklärt werden.\\
Die Darstellung der Messergebnisse in Kapitel~\ref{kap:ergebnisse} gliedert sich in sieben Teile.
Im ersten Teil (\ref{kap:überblick}) geht es um einen Überblick über die verwendeten Systeme und deren Anpassungsmöglichkeiten, gefolgt von einer Angabe der getesteten Konfiguration der Systeme in Kapitel
\ref{kap:getestetekonfiguration}.
Als erster Vergleich dient die Gegenüberstellung der Nutzung der Ressourcen des Nexys4-DDR \ac{fpga}-Board im Abschnitt~\ref{kap:synthese}, bei welcher die Ergebnisse für jedes System einzeln grafisch
und in Textform dargestellt wurden, um abschließend alle Systeme zu vergleichen.\\
Das gleiche Prinzip der Darstellung und des Vergleichens gilt für die Benchmarkergebnisse in den Kapiteln~\ref{kap:coremarktest},~\ref{kap:dhrystonetest} und~\ref{kap:ramspeedtest}.
Das Fazit, welches sich im Kapitel~\ref{kap:fazit} befindet, fasst die erreichten Ergebnisse nochmal zusammen und erschafft ein Gesamtbild. In diesem Kapitel wurden nochmals Besonderheiten der
Systeme erläutert, welche für einzelne Testergebnisse von Bedeutung waren.
Im Kapitel~\ref{ch:schluss} wird die Arbeit noch einmal zusammengefasst und ein kurzer Ausblick gegeben.\\
