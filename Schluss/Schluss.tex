\chapter{Zusammenfassung und Ausblick}\label{ch:schluss}
In dieser Arbeit wurden grundsätzliche Themen, welche mit eingebetteten Systemen zusammenhängen in Form der Grundlagen erläutert und zum Teil,
 zum besseren Verständnis, grafisch dargestellt. Des Weiteren wurden die verwendeten Benchmarks im Allgemeinen und in ihren Besonderheiten erklärt.\\
 Ferner wurde ein Überblick über die drei getestetem Systeme geschaffen, indem diese näher erläutert wurden, gefolgt von einer beschriebenen Konfiguration,
 an welche im Idealfall alle Systeme angepasst werden sollten. Anschließend wurden die erreichten Ergebnisse bezüglich ihrer Benchmark für jedes
 System dargestellt und ein direkter Vergleich erstellt und bewertet. Dieses Vorgehen endete in einem Fazit, in welchem verschiedene Zusammenhänge geschildert wurden.\\
Die in dem Fazit getroffenen Aussagen zum Thema der Anpassung und Optimierung bieten ein Potenzial für alle Systeme. Durch die erreichten Testergebnisse lassen sich etwaige
Schwächen eines Systems und gegebenenfalls Lösungsansätze erkennen. Das System der Professur für Technische Informatik an der Helmut-Schmidt-Universität / Universität Hamburg
der Bundeswehr, welches der ausschlaggebende Punkt für diese Arbeit war, konnte in einigen Beurteilungspunkten seine Leistungs zeigen, bietet jedoch Entwicklungspotenzial,
vor allem im Bezug auf die Nutzung der vorhanden Ressourcen eines \ac{fpga}. Um das System weiter zu optimieren und in der Leistung zu steigern, bietet beispielsweise eine
funktionsfähige \ac{fpu} enormes Potenzial, welche, neben der Optimierung des Ressourcenverbrauchs, als zukünftiges Ziel aus dieser Arbeit hervorgehen könnte.
